\documentclass[12pt]{article}

\usepackage[a4paper,margin=2cm]{geometry}
\usepackage{amsmath, amssymb, amsthm, amsfonts, tikz, algpseudocode}
\usepackage[plain]{algorithm}
\usepackage[framemethod=TikZ]{mdframed}
\definecolor{newblue}{rgb}{0.2,0.2,0.6}
\usepackage[colorlinks,allcolors=newblue]{hyperref}

\DeclareMathOperator*{\argmin}{arg\,min}
\DeclareMathOperator*{\eig}{eig}

\newcommand{\handout}[5]{
   \renewcommand{\thepage}{#1-\arabic{page}}
   \noindent
   \begin{center}
   \framebox{
      \vbox{
    \hbox to 5.78in { {\bf MATH 690: Topics in Probablity Theory} \hfill #2 }
       \vspace{4mm}
       \hbox to 5.78in { {\Large \hfill #5  \hfill} }
       \vspace{2mm}
       \hbox to 5.78in { {\it #3 \hfill #4} }
      }
   }
   \end{center}
   \vspace*{4mm}
}

\renewcommand{\paragraph}[1]{\medskip \noindent {\bf #1.}}

\newcommand{\lecture}[4]{\handout{#1}{#2}{Lecturer: #3}{Scribes: #4}{#1}}

%%%% TITLE

\title{Math 690: Topics in Data Analysis and Computation \\
Lecture notes for September 29, 2017}
\date{}

\author{Scribed by Dev Dabke and Andrew Cho}

%%%%%% FOUND AT
% https://texblog.org/2015/09/30/fancy-boxes-for-theorem-lemma-and-proof-with-mdframed/

%%%%%%%%%%%%%%%%%%%%%%%%%%%%%%
%Theorem
\newcounter{theo}[section]\setcounter{theo}{0}
\renewcommand{\thetheo}{\arabic{section}.\arabic{theo}}
\newenvironment{theo}[2][]{%
\refstepcounter{theo}%
\ifstrempty{#1}%
{\mdfsetup{%
frametitle={%
\tikz[baseline=(current bounding box.east),outer sep=0pt]
\node[anchor=east,rectangle,fill=blue!20]
{\strut Theorem~\thetheo};}}
}%
{\mdfsetup{%
frametitle={%
\tikz[baseline=(current bounding box.east),outer sep=0pt]
\node[anchor=east,rectangle,fill=blue!20]
{\strut Theorem~\thetheo:~#1};}}%
}%
\mdfsetup{innertopmargin=5pt,innerbottommargin=10pt,linecolor=blue!20,%
linewidth=2pt,topline=true,%
frametitleaboveskip=\dimexpr-\ht\strutbox\relax
}
\begin{mdframed}[]\relax%
\label{#2}}{\end{mdframed}}

%%%%%%%%%%%%%%%%%%%%%%%%%%%%%%
%Lemma

\newcounter{lem}[section]\setcounter{lem}{0}
\renewcommand{\thelem}{\arabic{section}.\arabic{lem}}
\newenvironment{lem}[2][]{%
\refstepcounter{lem}%
\ifstrempty{#1}%
{\mdfsetup{%
frametitle={%
\tikz[baseline=(current bounding box.east),outer sep=0pt]
\node[anchor=east,rectangle,fill=green!20]
{\strut Lemma~\thelem};}}
}%
{\mdfsetup{%
frametitle={%
\tikz[baseline=(current bounding box.east),outer sep=0pt]
\node[anchor=east,rectangle,fill=green!20]
{\strut Lemma~\thetheo:~#1};}}%
}%
\mdfsetup{innertopmargin=5pt,innerbottommargin=10pt,linecolor=green!20,%
linewidth=2pt,topline=true,%
frametitleaboveskip=\dimexpr-\ht\strutbox\relax
}
\begin{mdframed}[]\relax%
\label{#2}}{\end{mdframed}}

%%%%%%%%%%%%%%%%%%%%%%%%%%%%%%
%Proof

\newcounter{prf}[section]\setcounter{prf}{0}
\renewcommand{\theprf}{\arabic{section}.\arabic{prf}}
\newenvironment{prf}[2][]{%
\refstepcounter{prf}%
\ifstrempty{#1}%
{\mdfsetup{%
frametitle={%
\tikz[baseline=(current bounding box.east),outer sep=0pt]
\node[anchor=east,rectangle,fill=red!20]
{\strut Proof~\theprf};}}
}%
{\mdfsetup{%
frametitle={%
\tikz[baseline=(current bounding box.east),outer sep=0pt]
\node[anchor=east,rectangle,fill=red!20]
{\strut Proof~\thetheo:~#1};}}%
}%
\mdfsetup{innertopmargin=5pt,innerbottommargin=10pt,linecolor=red!20,%
linewidth=2pt,topline=true,%
frametitleaboveskip=\dimexpr-\ht\strutbox\relax
}
\begin{mdframed}[]\relax%
\label{#2}}{\end{mdframed}}

%%%%%%%%%%%%%%%%%%%%%%%%%%%%%%
%Remark
\newcounter{rmk}[section]\setcounter{rmk}{0}
\renewcommand{\thermk}{\arabic{section}.\arabic{rmk}}
\newenvironment{rmk}[2][]{%
\refstepcounter{rmk}%
\ifstrempty{#1}%
{\mdfsetup{%
frametitle={%
\tikz[baseline=(current bounding box.east),outer sep=0pt]
\node[anchor=east,rectangle,fill=yellow!50]
{\strut Remark~\thermk};}}
}%
{\mdfsetup{%
frametitle={%
\tikz[baseline=(current bounding box.east),outer sep=0pt]
\node[anchor=east,rectangle,fill=yellow!50]
{\strut Remark~\thermk:~#1};}}%
}%
\mdfsetup{innertopmargin=5pt,innerbottommargin=10pt,linecolor=yellow!50,%
linewidth=2pt,topline=true,%
frametitleaboveskip=\dimexpr-\ht\strutbox\relax
}
\begin{mdframed}[]\relax%
\label{#2}}{\end{mdframed}}

%%%%%%%%%%%%%%%%%%%%%%%%%%%%%%
%Definiton
\newcounter{dfn}[section]\setcounter{dfn}{0}
\renewcommand{\thedfn}{\arabic{section}.\arabic{dfn}}
\newenvironment{dfn}[2][]{%
\refstepcounter{dfn}%
\ifstrempty{#1}%
{\mdfsetup{%
frametitle={%
\tikz[baseline=(current bounding box.east),outer sep=0pt]
\node[anchor=east,rectangle,fill=purple!20]
{\strut Definition~\thedfn};}}
}%
{\mdfsetup{%
frametitle={%
\tikz[baseline=(current bounding box.east),outer sep=0pt]
\node[anchor=east,rectangle,fill=purple!20]
{\strut Definition~\thedfn:~#1};}}%
}%
\mdfsetup{innertopmargin=5pt,innerbottommargin=10pt,linecolor=purple!20,%
linewidth=2pt,topline=true,%
frametitleaboveskip=\dimexpr-\ht\strutbox\relax
}
\begin{mdframed}[]\relax%
\label{#2}}{\end{mdframed}}


\begin{document}
\lecture{Dimension Reduction}{September 29, 2017}{Xiuyuan Cheng}{Dev Dabke and Andrew Cho}

\section{Introduction}
The lecture covered
\begin{itemize}
	\item A wrap up of \textit{Topic 2: Dimension Reduction} by discussing convergence of eignemaps when $ p(x) $ is not uniform
	\item The start of \textit{Topic 3: Clustering}
\end{itemize}

\section{Dimension Reduction}
Recall the convergence of eigenmap
\begin{align*}
	&L_{n, \epsilon} \xrightarrow{n\to\infty} L_{\epsilon} \xrightarrow{\Sigma \to 0} L & \\
	&L_{\epsilon} f \xrightarrow{\epsilon \to 0} Lf &\text{for each } f
\end{align*}
This is a point-wise convergence operator and doesn't necessarily mean uniform convergence.
Rather, what we need is a convergence of the spectrum $ \eig{L_{\Sigma}} \to \eig{L} $. In essence, we seek $\sup { \| L_{\Sigma} f - L f \| } \to 0$ where $ f \in C^2 (M) $, $ \| f \|_2^2 = 1 $, and $\int{f(x)^2dp(x)=1}$. Unfortunately, these last two conditions are not always true.~\cite{belkin-2003}

\begin{dfn}[Heat Kernel]{dfn:heat}
Where $ t = \frac{1}{2} \epsilon $
\begin{align*}
	L_t &= \frac{I_{\alpha} - H_t}{t} + R_t \\
	H_t f(x) &= u(x, t)
\end{align*}
where the function $ u $ has constraints
\begin{align*}
 u_{t} &= -\Delta_M u \\
	u(x, 0) &= f(x)
\end{align*}
\end{dfn}
As a result, we have that the residual $ \| R_t \| $ can be controlled properly
which implies that $ \eig{L_t} = \eig{( \frac{I_d - H_t}{t} )} $
and $ H_t f = e^{-t \Delta_M} f $
\begin{rmk}[Exponential ODE]{rmk:ode}
$ y'(t) = -at \implies y = e^{-at} y(0) $.
\end{rmk}
Additionally
\[
\frac{1 - e^{-t\lambda_k}}{t} \xrightarrow{t \to 0} \lambda_k
\]
Anyways, note that
\[
H_t f = e^{-t \Delta_M} f
\]
such that $ \Delta_M : \{ \lambda_k, \psi_k \}_k $
and that $ H_t : \{ e^{-t \lambda_k}, \psi_k \}_k $
such that $ k = 1, \ldots, d $.
\\ \\
\begin{rmk}[$ p(x) $ Uniformity]{rmk:p-uniform}
	When $ p(x) $ is not uniform, then
	\[
	L_{n, \epsilon} \to L_{FK}
	\]
	where $ L_{FK} f = \Delta_M f - \nabla u \dot \nabla f $.
	and that
	\begin{align*}
		p(x) &= e^{-\frac{1}{2} u(x)} \\
		u(x) &= -2 \log{p(x)}
	\end{align*}
	by Fokker-Planck.
\end{rmk}
We have to perform a ``correction'' of density by defining a Weight Matrix $ W $ such that $ W_{ij} = k(x_i, x_j) $.
\begin{dfn}[Affinity Matrix]{dfn:affinity}
	Let
	\begin{align*}
		d_i &= \sum_{j} k(x_i, x_j) \\
		\widetilde{k} (x, y) &= \frac{k(x, y)}{\sqrt{d(x)} \sqrt{d(y)}} \\
		d(x) &= \int_M k(x, y) p(y) dy \\
	\end{align*}
	where $ d $ is the degree function.
\end{dfn}
In practice, we cannot take a continuous integral, so instead we compute
\[
d_R (x) = \frac{1}{n} \sum_{j = 1}^n k(x, x_j) \xrightarrow{n \to \infty} d(x)
\]
and we let
\[
\widetilde{W_{ij}} = \frac{W_{ij}}{\sqrt{d(x)} \sqrt{d(y)}}
\]
and so consider the eigenmap from $ \widetilde{W} $ instead of $ W $.
\begin{theo}[Convergence of $ L $]{theo:convergence}
	Given the matrix $ \widetilde{L}_{rw} = I - \widetilde{D}^{-1} \widetilde{W} $ where
	\[
	\widetilde{D}_{ij} = \sum_j \widetilde{W}_{ij}
	\]
	then
	\[
	\widetilde{L}_{n, \epsilon} \xrightarrow{n \to \infty \\ \epsilon \to \infty} \Delta_M
	\]
\end{theo}
\begin{prf}[Convergence of $ L $, Theorem~\ref{theo:convergence}]{prf:convergence}
	The proof is omitted, but as hint, note that $ \epsilon \to 0, \, d_{\epsilon} (x) \approx p(x) \cdot \text{constant} $.
	\\ \\
	Additionally, we can generalize this to a graph Laplacian with any $ 0 < \alpha < 1 $.
	The corrected kernel $\widetilde{k}$ above uses $\alpha = \frac{1}{2}$.
	Therefore, we write
	\[
	\widetilde{L}_{\alpha} = \frac{W_{ij}}{d_i^{\alpha} d_j^{\alpha}}
	\]
	Recall that $k_{\epsilon} (x, y) = e^{= \frac{\| x - y \|^2}{2 \epsilon}}$ and $d_{\epsilon} (x) = \int_M k_{\epsilon} (x, y) p(y) dy \approx p(x)$.
\end{prf}

\section{Topic 3: Clustering}
We start the discussion of our third topic on clustering by defining what the problem of clustering is.
\\ \\
Problem: given $ \{ x_i \}_{i = 1}^n $, find clusters.
These clusters may or may not have labels (\textit{supervised} vs. \textit{unsupervised} learning).
There are many possible definitions and models of clusters.
For example, we will consider two possible cases:
\begin{enumerate}
    \item given data points
    \item given graph, affinity matrix $W $ is $ n \times n $ where $ W_{ij} $ is the similarity of node $ i $ and $ j $
\end{enumerate}

\subsection{Case 1: With Data Points}
We will consider a better and precise formulation of ``clusters'' using a scheme of ``hard membership.''
\\ \\
\begin{dfn}[Cluster]{dfn:cluster}
	Given $ \{ x_i \}_{i = 1}^n $, find a partition of the vertices $ \mathcal{V} = \{ 1, \ldots, n \} $
	into disjoint subsets $ \mathcal{C} = \{ C_1, \ldots, C_k \} $
	such that
	\[
	\mathcal{V} = \bigcup_{C \in \mathcal{C}} C
	\]
	where ``disjoint'' means $ C_{l} \cap C_{l'} = \{ \emptyset \} \iff l \neq l' $.
	\\ \\
	We say that each $ C_{i} $ is the $ i^{th} $ cluster.
\end{dfn}
\begin{rmk}[Soft Membership]{rmk:soft}
	We can also consider some idea of ``soft membership.''
	In this case, we have some probability profile over each node such that
	$\mathbb{P} (\text{node}~ i \in C_l) = p_{i,l}$ with the constraint that $ \forall i,~ \sum_{l = 1}^k p_{il} = 1$
\end{rmk}
\begin{dfn}[$ k $-means]{dfn:k-means}
	We use the following algorithm~\cite{lloyd-1982}
	\begin{enumerate}
	    \item Seeding: Randomly generate ``centroids'' $ \{ \mu_1, \ldots, \mu_k \} = \mu $
	    \item Assignment: $ \forall i $ assign $ x_i $ to the closest centroid in $ \mu $ and this gives a partition $ \mathcal{C} $
	    \item Update of $ \mu $: for $ l = 1, \ldots, k $ we compute an updated $ \mu_l' $ where we let
	    \[
	    \mu_l' = \frac{1}{| C_{l} |} \sum_{i \in C_l} x_i
	    \]
	    and $ | C_{l} | $ is ``the cardinal number of the set $ C_l $.''
	\end{enumerate}
	After step $ 3 $, we repeat step $ 2-3 $ until we reach the stopping condition: $\| \mu_{\textsc{new}} - \mu_{\textsc{old}} \| < \delta$ for some tolerance level $ \delta $.
\end{dfn}
\begin{theo}[Optimality of $ k $-means]{theo:min-k-means}
	The process in Definition~\ref{dfn:k-means} solves the objective function
	\[
	\underset{\mu, C}{\operatorname{argmin}}  \sum_{l = 1}^k \sum_{i \in C_l} \| x_i - \mu_l \|^2
	\]
\end{theo}
\begin{rmk}[$ k $-means and $ k $-medians]{rmk:k-means-medians}
The squared $L^2$ norm $\| x_i - \mu_l \|_2^2$ gives the formulation of $k$-means. If using the (unsquared) $L^1$ norm $\| x_i - \mu_l \|_1$, it leads to the objective function of $\textit{k-medians}$. One can also remove the square, that is, using $\| x_i - \mu_l \|_2$ instead of $\| x_i - \mu_l \|_2^2$, which is a mixed $L^2$-$L^1$ norm.
\end{rmk}

\bibliography{clustering}{}
\bibliographystyle{alpha}

\end{document}
