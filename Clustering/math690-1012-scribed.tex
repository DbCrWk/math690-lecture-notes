\documentclass[12pt]{article}

\usepackage[a4paper,margin=2cm]{geometry}
\usepackage{amsmath, amssymb, amsthm, amsfonts, tikz, algpseudocode}
\usepackage[plain]{algorithm}
\usepackage[framemethod=TikZ]{mdframed}
\definecolor{newblue}{rgb}{0.2,0.2,0.6}
\usepackage{caption}
\usepackage{graphicx}
\usepackage{float}
\usepackage{listings}
\usepackage{color}
\usepackage[colorlinks,allcolors=newblue]{hyperref}
\usepackage{listings}
\lstset{
    language=R,
    basicstyle=\scriptsize\ttfamily,
    commentstyle=\ttfamily\color{gray},
    numbers=left,
    numberstyle=\ttfamily\color{gray}\footnotesize,
    stepnumber=1,
    numbersep=5pt,
    backgroundcolor=\color{white},
    showspaces=false,
    showstringspaces=false,
    showtabs=false,
    frame=single,
    tabsize=2,
    captionpos=b,
    breaklines=true,
    breakatwhitespace=false,
    title=\ttfamily\lstname,
    escapeinside={},
    keywordstyle={},
    morekeywords={}
}

\theoremstyle{plain}
\newtheorem*{theorem}{Theorem}
\newtheorem*{lemma}{Lemma}
\newtheorem*{claim}{Claim}
\newtheorem*{definition}{Definition}
\newtheorem*{corollary}{Corollary}
\newtheorem*{remark}{Remark}
\newtheorem*{proposition}{Proposition}
\DeclareMathOperator*{\argmin}{arg\,min}
\DeclareMathOperator*{\eig}{eig}
\DeclareMathOperator*{\diag}{diag}
\DeclareMathOperator*{\vol}{Vol}

\newcommand{\handout}[5]{
   \renewcommand{\thepage}{#1-\arabic{page}}
   \noindent
   \begin{center}
   \framebox{
      \vbox{
    \hbox to 5.78in { {\bf MATH 690: Topics in Probablity Theory} \hfill #2 }
       \vspace{4mm}
       \hbox to 5.78in { {\Large \hfill #5  \hfill} }
       \vspace{2mm}
       \hbox to 5.78in { {\it #3 \hfill #4} }
      }
   }
   \end{center}
   \vspace*{4mm}
}

\renewcommand{\paragraph}[1]{\medskip \noindent {\bf #1.}}

\newcommand{\lecture}[4]{\handout{#1}{#2}{Lecturer: #3}{Scribes: #4}{#1}}

%%%% TITLE

\title{Math 690: Topics in Data Analysis and Computation \\
Lecture notes for October 12, 2017}
\date{}

\author{Scribed by Dev Dabke and Andrew Cho}

%%%%%% FOUND AT
% https://texblog.org/2015/09/30/fancy-boxes-for-theorem-lemma-and-proof-with-mdframed/

%%%%%%%%%%%%%%%%%%%%%%%%%%%%%%
%Theorem
\newcounter{theo}[section]\setcounter{theo}{0}
\renewcommand{\thetheo}{\arabic{section}.\arabic{theo}}
\newenvironment{theo}[2][]{%
\refstepcounter{theo}%
\ifstrempty{#1}%
{\mdfsetup{%
frametitle={%
\tikz[baseline=(current bounding box.east),outer sep=0pt]
\node[anchor=east,rectangle,fill=blue!20]
{\strut Theorem~\thetheo};}}
}%
{\mdfsetup{%
frametitle={%
\tikz[baseline=(current bounding box.east),outer sep=0pt]
\node[anchor=east,rectangle,fill=blue!20]
{\strut Theorem~\thetheo:~#1};}}%
}%
\mdfsetup{innertopmargin=5pt,innerbottommargin=10pt,linecolor=blue!20,%
linewidth=2pt,topline=true,%
frametitleaboveskip=\dimexpr-\ht\strutbox\relax
}
\begin{mdframed}[]\relax%
\label{#2}}{\end{mdframed}}

%%%%%%%%%%%%%%%%%%%%%%%%%%%%%%
%Lemma

\newcounter{lem}[section]\setcounter{lem}{0}
\renewcommand{\thelem}{\arabic{section}.\arabic{lem}}
\newenvironment{lem}[2][]{%
\refstepcounter{lem}%
\ifstrempty{#1}%
{\mdfsetup{%
frametitle={%
\tikz[baseline=(current bounding box.east),outer sep=0pt]
\node[anchor=east,rectangle,fill=green!20]
{\strut Lemma~\thelem};}}
}%
{\mdfsetup{%
frametitle={%
\tikz[baseline=(current bounding box.east),outer sep=0pt]
\node[anchor=east,rectangle,fill=green!20]
{\strut Lemma~\thetheo:~#1};}}%
}%
\mdfsetup{innertopmargin=5pt,innerbottommargin=10pt,linecolor=green!20,%
linewidth=2pt,topline=true,%
frametitleaboveskip=\dimexpr-\ht\strutbox\relax
}
\begin{mdframed}[]\relax%
\label{#2}}{\end{mdframed}}

%%%%%%%%%%%%%%%%%%%%%%%%%%%%%%
%Proof

\newcounter{prf}[section]\setcounter{prf}{0}
\renewcommand{\theprf}{\arabic{section}.\arabic{prf}}
\newenvironment{prf}[2][]{%
\refstepcounter{prf}%
\ifstrempty{#1}%
{\mdfsetup{%
frametitle={%
\tikz[baseline=(current bounding box.east),outer sep=0pt]
\node[anchor=east,rectangle,fill=red!20]
{\strut Proof~\theprf};}}
}%
{\mdfsetup{%
frametitle={%
\tikz[baseline=(current bounding box.east),outer sep=0pt]
\node[anchor=east,rectangle,fill=red!20]
{\strut Proof~\thetheo:~#1};}}%
}%
\mdfsetup{innertopmargin=5pt,innerbottommargin=10pt,linecolor=red!20,%
linewidth=2pt,topline=true,%
frametitleaboveskip=\dimexpr-\ht\strutbox\relax
}
\begin{mdframed}[]\relax%
\label{#2}}{\end{mdframed}}

%%%%%%%%%%%%%%%%%%%%%%%%%%%%%%
%Remark
\newcounter{rmk}[section]\setcounter{rmk}{0}
\renewcommand{\thermk}{\arabic{section}.\arabic{rmk}}
\newenvironment{rmk}[2][]{%
\refstepcounter{rmk}%
\ifstrempty{#1}%
{\mdfsetup{%
frametitle={%
\tikz[baseline=(current bounding box.east),outer sep=0pt]
\node[anchor=east,rectangle,fill=yellow!50]
{\strut Remark~\thermk};}}
}%
{\mdfsetup{%
frametitle={%
\tikz[baseline=(current bounding box.east),outer sep=0pt]
\node[anchor=east,rectangle,fill=yellow!50]
{\strut Remark~\thermk:~#1};}}%
}%
\mdfsetup{innertopmargin=5pt,innerbottommargin=10pt,linecolor=yellow!50,%
linewidth=2pt,topline=true,%
frametitleaboveskip=\dimexpr-\ht\strutbox\relax
}
\begin{mdframed}[]\relax%
\label{#2}}{\end{mdframed}}

%%%%%%%%%%%%%%%%%%%%%%%%%%%%%%
%Definiton
\newcounter{dfn}[section]\setcounter{dfn}{0}
\renewcommand{\thedfn}{\arabic{section}.\arabic{dfn}}
\newenvironment{dfn}[2][]{%
\refstepcounter{dfn}%
\ifstrempty{#1}%
{\mdfsetup{%
frametitle={%
\tikz[baseline=(current bounding box.east),outer sep=0pt]
\node[anchor=east,rectangle,fill=purple!20]
{\strut Definition~\thedfn};}}
}%
{\mdfsetup{%
frametitle={%
\tikz[baseline=(current bounding box.east),outer sep=0pt]
\node[anchor=east,rectangle,fill=purple!20]
{\strut Definition~\thedfn:~#1};}}%
}%
\mdfsetup{innertopmargin=5pt,innerbottommargin=10pt,linecolor=purple!20,%
linewidth=2pt,topline=true,%
frametitleaboveskip=\dimexpr-\ht\strutbox\relax
}
\begin{mdframed}[]\relax%
\label{#2}}{\end{mdframed}}


\begin{document}
\lecture{Clustering}{October 12, 2017}{Xiuyuan Cheng}{Dev Dabke and Andrew Cho}

\section{Introduction}
The lecture covered the following on the consistency of spectral clustering
\begin{itemize}
	\item Setup of $\mathcal{L}_{un}$ and $\mathcal{L}_n$ and their limit operators $U$ and $U'$
	\item First $r$ spectral convergence
\end{itemize}

\section{Consistency of Spectral Clustering}

This section is a high-level overview of the ideas presented by von Luxburg, including the details of the proofs.~\cite{luxburg-2008}

\subsection{The Problem Setup}

Suppose $ X_{i} \sim P $ where $ P $ is some distribution on $ \Omega \subset \mathbb{R}^{D} $.
$ W_{ij} $ is the affinity matrix where, as an example,
\[
W_{ij} = e^{\frac{-|x_i - x_j|^2}{\varepsilon}}
\]
where $ \varepsilon > 0 $.
\\ \\
As $ n \to \infty $, we want to show the convergence of the graph Laplacian $ \mathcal{L} $.
\begin{enumerate}
  \item $ \mathcal{L}_{n} = D - W $ where $ D_{ij} = \sum_{j = 1}^{n} W_{ij} $ ($  \to U $), i.e.\ unnormalized
  \item $ \mathcal{L}_{n}' = D^{-\frac{1}{2}} (D - W) D^{-\frac{1}{2}} $ where ($  \to U' $), i.e.\ symmetric
  \item $ \mathcal{L}_{n}'' = D^{-1} (D - W) = I - P $, i.e.\ random walk
\end{enumerate}

\subsection{Limit operators $U$ and $U'$}
We construct linear limit operators $U$ and $U'$ on $C(X)$ which are the limit of the discrete operators $\mathcal{L}_n$ and $\mathcal{L}_n'$. We prove that the first "r" eigenvectors of the discrete operators converge to eigenfuction  of the limit operators.
\begin{dfn}[Limit Operators]{dfn:limit-operators}
	We define $ U $ as
	\begin{align*}
	  U &: C(\Omega) \to C(\Omega) \\
	  Uf(x) &= f(x)d(x) - \int k(x, y) f(y) dP(y)
	\end{align*}
	where
	\begin{align*}
	  dP(x) &= p(x)dx \\
	  dx &= \int k(x, y) dP(y) \\
	  x \in \Omega
	\end{align*}
\end{dfn}
\begin{theo}[$ U' $]{theo:u-prime}
	\[
	U' f(x) = f(x) - \int{ \frac{ k(x, y) }{ \sqrt{ d(x) } \sqrt{ d(y) } } f(y) dP(y) }
	\]
\end{theo}

\begin{prf}[$ U' $, Theorem~\ref{theo:u-prime}]{prf:u-prime}

  \begin{align*}
    (D^{-\frac{1}{2}} W D^{-\frac{1}{2}})_{ij} &= \frac{1}{\sqrt{D_{ii}}} W_{ij} \frac{1}{\sqrt{D_{jj}}} \\
    &= \frac{ \frac{1}{n} k(x_i, x_j) }{ \sqrt{ \frac{1}{n} \sum_{j'} k(x_i, x_j') } \sqrt{ \frac{1}{n} \sum_{j'} k(x_j, x_j') } } \\
    &\approx \frac{ \frac{1}{n} k(x_i, x_j) }{ \sqrt{d(x_i)} \sqrt{d(x_j)} } \\
  \end{align*}

\end{prf}

\subsection{First $ r $ spectral convergence}

Let's discuss what ``first $ r $ spectral convergence'' means.
$ M_n $ first $ r $ spectral convergence to T if first $ r $ eigenvalues of $ M_n $ converge to those of $ T $, and the associated eigenvectors converge to the eigenfunctions of $ T $, the first smallest $ r $ eigenvalues.

\begin{theo}[Convergence]{theo:convergence}
For fixed $r > 0$, $n \rightarrow \infty$, and mild conditions,
	\begin{enumerate}
	\item ($\textbf{Unnormalized}$) $\mathcal{L}_n$ first r spectral converge to $ U $ if the first r eigenvalues of $ U $ lie outside of the range of the degree function $ d(x) $. We need extra constraints for convergence since $ U $ might coincide with the range of $ d(x) $.
	\item ($\textbf{Normalized Symmetric}$) $\mathcal{L}_n'$ first r spectral converge to the operator $ U' $.
	\end{enumerate}
\end{theo}

\begin{prf}[Convergence, Theorem~\ref{theo:convergence}]{prf:convergence}
(Case 2 $\mathcal{L}_{sym}$) Have $M_n$ converge to $T$ where
	\begin{align*}
		M_n &: \mathbb{R}^n \rightarrow \mathbb{R}^n \\
		T &: C(\Omega) \rightarrow C(\Omega) \\
		M_n &: I-D^{\frac{-1}{2}}WD^{\frac{-1}{2}} \\
		T &= U'
	\end{align*}

	\begin{align*}
	Tf(x) &= f(x) - \int h(x,y)f(y)dP(y) \\
	&= f(x) - \int h(x,y)f(y)dP_n(y) \\
	&= f(x) - \frac{1}{n} \sum_{i=1}^n h(x,x_i)f(x_i)
	\end{align*}
	where
	\begin{align*}
	&h(x,y) = \frac{k(x,y)}{\sqrt{d(x)}\sqrt{d(y)}} \\
	&dP_n(x) = \frac{1}{n} \sum_{i=1}^n \delta_{x_i}(x) dx
	\end{align*}

\end{prf}

\begin{lem}[Spectral Equivalence between $M_n$ and $T_n$]{lem:spectral-equivalence}
	\begin{enumerate}
	\item If $T_n \varphi = \lambda \varphi$, let $v \in \mathbb{R}^n, i = 1,...,n v_i = \varphi(x_i)$, then $M_nv = \lambda v$.
	\item If $M_nv = \lambda v$ and $\lambda \neq 1$, then let $\varphi = \frac{\frac{1}{n} \sum h(x,x_j)v_j}{1-\lambda}$ and so then $T_n\varphi = \lambda\varphi$.
	\end{enumerate}
\end{lem}

\begin{lem}[Spectral Convergence]{lem:spectral-convergence}
	Replacing $ dP_n(y) $ to be $ dP(y) $, $ T_n $ spectral converges to $ T $.
\end{lem}

\begin{prf}[Spectral Convergence, Lemma~\ref{lem:spectral-convergence}]{prf:spectral-convergence}
For all $f$, $T_n \rightarrow Tf$ by the Law of Large Numbers.
$ \|T_nf - Tf\|_{\inf} \rightarrow 0 $ simultaneously for ``sufficiently many'' $ f $ such that for each eigenvalue of $ T (\lambda \neq 1) $, the associated eigenvalue of $T_n$ converge to $\lambda$ and the associated eigenfunction of $T_n$ converges.
In other words,
$ T_n \varphi_n = \lambda_n \varphi_n $ so $ \lambda_n \rightarrow$asymptotically
$T\varphi = \lambda \varphi$ and $ \| \varphi_n- \varphi \|_{\infty} \rightarrow 0$
\end{prf}

\bibliography{clustering}{}
\bibliographystyle{alpha}

\end{document}
